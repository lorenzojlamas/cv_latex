\documentclass[11pt,a4paper,sans]{moderncv}
\moderncvstyle{classic}
\moderncvcolor{black}
\usepackage[utf8]{inputenc}
\usepackage[spanish,es-lcroman, activeacute]{babel} 


\usepackage[scale=0.76]{geometry}
\setlength{\hintscolumnwidth}{3cm}
\firstname{Lorenzo José}
\lastname{Lamas}
\title{Tech Manager}
\email{lorenzojlamas@gmail.com}
\homepage{https://www.linkedin.com/in/lorenzojlamas/}
\extrainfo{CABA, Argentina}
%----------------------------------------------------------------------------------
%            contenido
%----------------------------------------------------------------------------------
\begin{document}
\maketitle
\section{Resumen}

Tech Manager. 7 años de experiencia desarrollando soluciones de punta a punta. Recientemente especializado en escalar soluciones no code.
En los últimos años ejercí roles de liderazgo, desde el mentoreo y acompañamiento del plan de carrera del equipo, abarcando también gestión de stakeholders y participación en definiciones de producto (OKRS, Roadmaps, etc)
Trabajé en el desarrollo de diversos chatbots, integración con CRMs, I+D

\section{Hitos}
Dentro de Properati (Olx - LifullConnect) trabajé en el diseño de soluciones tecnológicas para apuntalar la performance de una unidad de negocio de venta premium. Desarrollé distintas iniciativas tanto code como no code, en diferentes etapas del ciclo de vida del cliente.(Hubspot, Airtable, Zappier, Whatsapp, Wolkbox, etc)
\\ \\
Desde el equipo de Telefónica Argentina diseñé e implementé de principio a fin el bot de la empresa que funciona actualmente en Argentina y varios países de la región. Desde el equipo de desarrollo se trabajó en conjunto con un equipo de NLP para productivizar los modelos de inteligencia artificial.

\section{Experiencia laboral}
\cventry{02/2022--11/2022}{Tech Manager}{Properati Argentina - LifullConnect}{}{}
{Tecnolog\'ias: \texttt{NodeJs, Express, Typescript, GCP, Firebase, docker (Kubernetes), Python, Postgres}}
\cventry{02/2021--01/2022}{Tech Lead}{Properati Argentina - Grupo OLX}{}{}
{Tecnolog\'ias: \texttt{NodeJs, Express, Typescript, GCP, Firebase, docker (Kubernetes), Python, Postgres}}
\cventry{04/2019--01/2021}{Desarrollador Sr}{Grupo Telefonica Argentina}{}{}
{Tecnolog\'ias: \texttt{NodeJs, Express, Chai, BotFramework, OpenApi, Azure, docker, AzureDevOps, Git, Python}}

\cventry{03/2018--03/2019}{Desarrollador I+D}{Practia Global}{}{}
{ Investigación y pruebas de concepto de nuevas tecnologías, especialmente en IA. Tecnolog\'ias: \texttt{openCV, face recognition, RASA, Luis, BotFramework, Azure, Jenkins, GitLab, VueJs, Angular, FLask, Docker, Proxmox.}}

\cventry{12/2013--02/2017}{Técnico de grupos electrógenos}{Aggreko Argentina}{}{}
{Relevado, conexión, configuración de grupos electrógenos, durante emergencias y en plantas.}

\cventry{06/2012--12/2013}{Técnico en automatismo industrial}{GM2Dev}{}{}
{ Desarrollo, fabricación y puesta en marcha de equipamiento de automatización industrial. Tecnolog\'ias: \texttt{PLC, HMI, Sistemas SCADA (Redes 232,485, ethernet), siemens, Delta}}


\section{Habilidades }
TypeScript, NodeJs, Python, Flask, Docker, GCP, Azure, BotFramework, AzureDeOps, git, TDD. Trunk base Development\\ 
Scrum, agile, liderazgo, mentoreo
\section{Idiomas}
\cvitemwithcomment{Ingl\'es}{Cursando B1}{}

\section{Formaci\'on acad\'emica}
\cventry{2016–2018 (incompleta)}{Licenciatura en Ciencias Matemáticas}{Facultad de Ciencia Exactas y Naturales - UBA}{}{\textit{Universitario}}{Materias aprobadas: Análisis Matemático y Análisis de datos}
\cventry{2015–2016}{CBC - Licenciatura en Ciencias Matemáticas}{Facultad de Ciencia Exactas y Naturales - UBA}{}{\textit{Universitario}}{}
\cventry{2006--2012}{Técnico electrónico}{E.T. Nº28 D.E. 10 - República Francesa}{}{\textit{Secundario}}{}

\section{Cursos y capacitaciones}
\cvlistitem{ Construcción de Software Robusto con TDD - 10Pines (2022)}
\cvlistitem{ Diseño Avanzado de Software con Objetos I - 10Pines (2022)}
\cvlistitem{ Oyente en materia: Ingeniería de Software - UNTreF (2021)}
\cvlistitem{ Oyente en materia: METODOLOGÍA DE LA CONDUCCIÓN DE EQUIPOS DE TRABAJO - UTN (2021)}
\cvlistitem{Azure Fundamentals - Grupo Telefónica (2020)}
\cvlistitem{Machine learning \& Data analytics - Grupo Telefonica (2019)}
\cvlistitem{Primera Jornada de Capacitación en Herramientas TIC Aplicadas a la Enseñanza de la Física - UTN (2011)}
\cvlistitem{Java - Nativos digitales (2011)}

\end{document}
